%%%%%%%%%%%%%%%%%%%%%%%%%%%%%%%%%%%%%%%%%
% Medium Length Professional CV
% LaTeX Template
% Version 2.0 (8/5/13)
%
% This template has been downloaded from:
% http://www.LaTeXTemplates.com
%
% Original author:
% Trey Hunner (http://www.treyhunner.com/)
%
% Important note:
% This template requires the resume.cls file to be in the same directory as the
% .tex file. The resume.cls file provides the resume style used for structuring the
% document.
%
%%%%%%%%%%%%%%%%%%%%%%%%%%%%%%%%%%%%%%%%%

%----------------------------------------------------------------------------------------
%	PACKAGES AND OTHER DOCUMENT CONFIGURATIONS
%----------------------------------------------------------------------------------------

\documentclass{resume} % Use the custom resume.cls style
\usepackage{color}
\renewcommand{\familydefault}{\rmdefault}
%color packages for timeline graph
\definecolor{dark}{RGB}{160, 64, 0}





\usepackage[left=0.5in,top=0.6in,right=0.5in,bottom=0.6in]{geometry} % Document margins
\usepackage[pdftex]{hyperref}
\newcommand{\tab}[1]{\hspace{.2667\textwidth}\rlap{#1}}
\newcommand{\itab}[1]{\hspace{0em}\rlap{#1}}

\hypersetup{
	colorlinks=true,
	linkcolor=blue,
	filecolor=dark,
	citecolor=dark,    
	urlcolor=dark,
}

\begin{document}
\rmfamily
\vspace{.75em}
\vspace{.75em}
\begin{tabular*}{\textwidth}{l@{\extracolsep{\fill}}r}
  \textbf{\href{https://anjalipverma.github.io/}{\LARGE {\color{dark}Anjali P. Verma}}} & {} \vspace{.5em} \\
  {University of Texas at Austin} &{}\\
  {Department of Economics} &  Email: \href{mailto:anjali.priya@utexas.edu}{anjali.priya@utexas.edu}\\
  {2225 Speedway C3100} & Website: \href{https://anjalipverma.github.io/}{anjalipverma.github.io}  \\
  {Austin, Texas - 78712} & Phone: +1-512-584-3248 
\end{tabular*}


%----------------------------------------------------------------------------------------
%	EDUCATION SECTION
%----------------------------------------------------------------------------------------

\begin{rSection}{Education}

{\bf University of Texas at Austin} \hfill {2018 - 2022 (Expected)} \\ 
Ph.D. in Economics \hfill {}

{\bf University of Texas at Austin} \hfill {2016 - 2018} \\
 M.S. in Economics \hfill {}

{\bf Delhi School of Economics, University of Delhi} \hfill {\ 2012 - 2014} \\ 
M.A. in Economics\hfill {}

{\bf Miranda House College, University of Delhi} \hfill { 2012 - 2014}\\ 
B.A. in Economics \hfill {}
\end{rSection}
%----------------------------------------------------------------------------------------
%	TECHNICAL STRENGTHS SECTION
%----------------------------------------------------------------------------------------

\begin{rSection}{References}
\begin{minipage}{0.5\linewidth}
{\bf Stephen J. Trejo (Co-chair)}\\
Department of Economics\\
University of Texas at Austin\\
+1 (512)-475-8512\\
\href{mailto:trejo@austin.utexas.edu}{trejo@austin.utexas.edu}\\
\end{minipage}
\hfill
\begin{minipage}{0.5\linewidth}
{\bf Richard Murphy (Co-chair)} \\
Department of Economics\\
University of Texas at Austin\\
+1 (512)-400-8068\\
\href{mailto:richard.murphy@austin.utexas.edu}{richard.murphy@austin.utexas.edu}\\
\end{minipage}

\begin{minipage}{0.5\linewidth}
{\bf Tom Vogl}\\
Department of Economics\\
University of California at San Diego\\
+1 (914)-582-2947\\
\href{mailto:tvogl@ucsd.edu}{tvogl@ucsd.edu}\\
\end{minipage}
\hfill
\begin{minipage}{0.5\linewidth}
{\bf Nishith Prakash} \\
Department of Economics\\
University of Conneticut\\
+1 (832)-474-6341\\
\href{mailto:nishith.prakash@uconn.edu}{nishith.prakash@uconn.edu}\\
\end{minipage}

\end{rSection}

%----------------------------------------------------------------------------------------
%	WORK EXPERIENCE SECTION
%----------------------------------------------------------------------------------------

\begin{rSection}{Research Fields}\itemsep -2pt

{\bf Primary:} Labor Economics, Development Economics\\
{\bf Secondary:} Economics of Education, Environmental Economics
%-----------------------------------------------

\end{rSection}


%	EXAMPLE SECTION
%----------------------------------------------------------------------------------------
\begin{rSection}{Past Employment} \itemsep -2pt
\begin{rSubsection}{Lecturer (Teaching Fellow), University of Delhi}{2015-2016}{}{}
\item[] \quad Intermediate Microeconomics, Development Economics, Business Economics
\end{rSubsection}
\end{rSection}


\begin{rSection}{Teaching and Research Experience} \itemsep -2pt
\begin{rSubsection}{Teaching Assistant, University of Texas at Austin}{}{}{}
\item[] \quad Introduction to Econometrics \hfill{2019-2021}
\item[] \quad Microeconomic Theory \hfill{2017-2019}
\item[] \quad Introduction to Macroeconomics\hfill{2017}
\item[] \quad Introduction to Microeconomics \hfill{2016}
\end{rSubsection}

\begin{rSubsection}{Research Assistant, University of Texas at Austin}{}{}{}
\item[] \quad Research Assistant, Prof. Sandra E. Black \hfill{2017-2018}
\item[] \quad Reserach Assistant, Prof. Kishore Gawande\hfill{2019}
\end{rSubsection}
\end{rSection}

%----------------------------------------------------------------------------------------
\begin{rSection}{Working Papers}
\vspace{.5em}
\href{https://anjalipverma.github.io/files/JMP\%20-\%20Anjali\%20P\%20Verma\%20-\%20UT\%20Austin.pdf}{\bf {\color{dark}Disruptive Interactions: Long-run Peer Effects of Disciplinary Schools}} {\bf (Job Market Paper)}\\Joint with A. Yonah Meiselman

Evidence suggests that exclusionary discipline such as temporary removals to disciplinary alternative schools has an adverse impact on students' long-run outcomes. This paper examines the role of disruptive peer effects at disciplinary alternative schools in impacting the future removal, educational attainment, and labor market outcomes of students placed at these schools. To study this, we use the restricted administrative records of all high-school students in Texas with a disciplinary placement between 2004 to 2018. Using the fact that a large number of regular schools feed disruptive students into a single disciplinary alternative school, we exploit the over-time variation in peer composition within a disciplinary school to estimate the causal effects of peers' disruptiveness on students’ outcomes. Our results show that having a peer group with higher average disruptiveness at the disciplinary school leads to 1) an increase in students' subsequent disciplinary removals 2) decline in their educational attainment (lower high-school graduation, college enrollment, and college graduation), and 3) decline in their adult employment and earnings ($\sim$ 8\% or 1272 USD decline in annual earnings at age 27). These results highlight the need to examine exclusionary discipline policies and adopt approaches that can mitigate the adverse effects of peers at disciplinary schools.
\vspace{.7em}


\href{https://anjalipverma.github.io/files/CleanEnergyAccess_V2.pdf}{\bf {\color{dark}Clean Energy Access: Gender Disparity, Health and Labor Supply} }\\Joint with Imelda, Conditional Acceptance, {\bf Economic Journal}

This paper studies the impact of clean energy access on adult health and labor supply with an emphasis
on its heterogeneity by gender. It exploits a nationwide clean cooking program in Indonesia that led
to plausibly exogenous variation in the timing of households’ access to cleaner energy, and large-scale
switching from a dirty fuel, kerosene, to a much cleaner cooking fuel, liquid petroleum gas. Using rich
longitudinal survey data from the Indonesia Family Life Survey and the staggered structure of the
program rollout, we find that access to clean cooking improves women's lung capacity, particularly
among those who are responsible for household work. Given that dirty cooking fuels are often one of
the biggest sources of indoor air pollution, we show that the reduction in pollution exposure is likely
the main channel. Furthermore, we find an increase in the labor supply of women, on both intensive
and extensive margins, which is consistent with women's increased domestic productivity allowing
them to work more hours.
\vspace{.7em}

\href{https://anjalipverma.github.io/files/VermaAnjali_AlimonyLaborSupply.pdf}{\bf{\color{dark} Female Labor Supply Response to Alimony: Evidence from Massachusetts}}\\ { Under Review}

This paper studies the labor supply response of women to changes in expected alimony. Using an alimony law change in the US that significantly reduced the post-divorce alimony support among women, I first show that this led to an increase in divorce probability. Second, consistent with the theoretical prediction from a simple model of labor supply, the reform led to an increase in the female labor force participation, with a larger increase among ever-married and more educated samples of women. As a result, the average female wage income increased after the reform. However, while labor supply increased, I show that most of this increase was concentrated in part-time employment, which may not be sufficient to compensate for the expected loss in alimony income. I estimate a net loss of \$40,621 in PDV of lifetime income due to the reform. In light of the recent movement in the US to reform alimony laws, these findings are pertinent to understand its implications on women's labor supply and economic well-being.\\

\href{https://anjalipverma.github.io/files/Heat_Technology_LaborProductivity.pdf}{\bf {\color{dark}Can Technology Mitigate the Impact of Heat on Labor Productivity? Evidence from India}}\\Joint with Anna Custers, Bhavani P. Kasina and Deepak Saraswat

This paper analyses the role of technology in reducing heat-induced labor productivity losses. For this, we use a field experiment in India which randomized the use of productivity-augmenting digital mode versus classic paper-and-pen mode for conducting 2000 household surveys. Combining this experimentally induced variation in survey mode with day-to-day variation in temperature, we estimate the impact of survey mode on surveyor productivity as temperature rises. We find that as temperature rises and working conditions start to deteriorate, using digital-mode results in 5 percent higher surveyor-productivity compared to paper surveys. These relative productivity gains are mainly concentrated on extremely hot days - where the adverse impact of heat is likely at its peak. We show that these impacts are not driven by differences in characteristcis of surveyor or respondents, thereby pointing to the role of technology in reducing the adverse effects of heat.\\

\end{rSection}


%-----------------------------------------------------------------
%----------------------------------------------------------------------------------------
\begin{rSection}{Selected Work in Progress}
\vspace{.5em}
{\bf Exclusionary Discipline: Impact of Student Removal to Disciplinary Alternative Programs }\\Joint with A. Yonah Meiselman 
\vspace{.7em}

{\bf To Apply or Not to Apply: Impact of Class Rank on College Application Choices}

\end{rSection}


%--------------------------------------------------------------

\begin{rSection}{Professional Activities}
\vspace{.56em}
\begin{rSubsection}{Conference Presentations}{}{}{}
\item[] Southern Economic Association (scheduled)  \hfill{2021}
\item[] APPAM Seminar Series \hfill{2021}
\item[] Southern Economic Association  \hfill{2020} 
\item[] Population Association of America, Washington DC (event canceled) \hfill{2020}
\item[] 15th Annual Conference on Economic Growth and Development, ISI Delhi \hfill{2019}
\item[] NEUDC, Northwestern University \hfill{2019}
\end{rSubsection}
\end{rSection}


%--------------------------------------------------------------

\begin{rSection}{Scholarships, and Fellowships
}
\begin{rSubsection}{}{}{}{}

\item[]Professional Development Fellowship, University of Texas at Austin \hfill{2021}
\item[]Professional Development Fellowship, University of Texas at Austin \hfill{2020}
\item[]Summer Research Fellowship, University of Texas at Austin  \hfill{2019} 
\item[]Professional Development Fellowship, University of Texas at Austin \hfill{2019}
\item[] Departmental Fellowship, University of Texas at Austin\hfill{2016}
\item[] Pradeep Gupta Memorial Scholarship, University of Delh \hfill{2012-13}
\end{rSubsection}
\end{rSection}




%--------------------------------------------------------------

\begin{rSection}{Programming skills}

{\bf Proficient:} Stata, LaTeX, MS Office
\\ {\bf Basic:} Python, R \\

\end{rSection}

%----------------------------------------------------------------------------------------
\begin{rSection}{Other Information} 
{\bf Citizenship:} Indian (F1 Visa)
\\ {\bf Languages:} English (fluent), Hindi (fluent)  \\

\end{rSection}

\end{document}
